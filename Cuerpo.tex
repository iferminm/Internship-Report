\documentclass[a4paper, 12pt]{article}
\usepackage[utf8]{inputenc}
\usepackage[spanish]{babel}
\usepackage{fancyhdr}
\usepackage{setspace}
\usepackage[scaled]{helvet}
\usepackage{makeidx}

\setlength{\parskip}{\baselineskip}

\pagestyle{fancy}
\lhead{}
\chead{}
\rhead{Israel Fermín Montilla}
\lfoot{}
\cfoot{\thepage}
\rfoot{}
\renewcommand{\headrulewidth}{0.4pt}
\renewcommand{\footrulewidth}{0.4pt}

\begin{document}
\lhead{Índice}
\tableofcontents

\newpage

\section{Sinopsis}
\lhead{Sinopsis}
El presente informe, describe todos los detalles de la pasantía realizada por el Br. Israel Fermín Montilla en la empresa Vauxoo, desarrollando un Módulo para la herramienta de Planificación de Recursos Empresariales (ERP) llamada OpenERP, utilizando OpenObject como marco de trabajo y el lenguaje de programación Python, para la empresa mexicana Clima Perfecto (Cliper), la cual se dedica a proveer servicio de Mantenimiento e Instalación de equipos de Aire Acondicionado y Refrigeración, a través de contratos de póliza. El módulo desarrollado, debía soportar el proceso de facturación automática para dichos contratos.

El desarrollo del módulo surge de la necesidad en Cliper de cambiar su actual sistema ERP, el cual es un desarrollo interno basado en SQL Server 2005. El modelo de datos usado por el sistema antiguo será mostrado en el desarrollo de este informe. Este sistema, sólo provee información acerca de las pólizas activas mas no automatiza los procesos como tal. Con la migración a OpenERP, de busca mejorar el estado actual en cuanto a Sistemas Administrativos en Cliper, mediante la automatización del sistema de contabilidad, administración de recursos humanos y gestión automática de pólizas de mantenimiento de aires acondicionados, si bien la situación no se puede considerar "mala" o "deficiente" en cuanto a tecnología, pues si existía un sistema para gestionar la información de clientes, proveedores, personal y contratos, esta se podía mejorar mediante la automatización de procesos (o subprocesos) críticos de la empresa y esto es exactamente lo que se desea lograr a través de OpenERP.

En este informe se describirá todo el desarrollo del módulo de gestión y facturación automática para pólizas de mantenimiento de aires acondicionados y equipos de refrigeración, desde el comienzo, hasta la entrega final del módulo. 

\newpage
\section{Planteamiento del Problema}
\lhead{Planteamiento del Problema}
La situación actual de Cliper, en cuanto a sistemas administrativos, no puede decirse que era mala o deficiente, sino más bien regular y mejorable, pues efectivamente existe un sistema para gestionar Proveedores, Clientes, Contratos, Equipos y todas las entidades que es necesario controlar para el buen funcionamiento del negocio de Cliper. El sistema está basado en Microsoft SQL Server 2005, lo cual significa que únicamente puede correr sobre Sistema Operativo Windows, lo que, además, implica el pago de ciertas licencias para poder utilizar el software de manera legal. También, el sistema fue producto de un desarrollo interno de la empresa, lo cual, si bien garantiza que se adapte bastante bien al esquema de negocio que maneja Cliper, resulta difícil la escalabilidad y la continuidad del proyecto pues, sólo una persona, conoce a profundidad el trabajo realizado y el funcionamiento del sistema.

Lo que se puede concluir de la situación actual es que más temprano que tarde, haría falta escalar el sistema ya que Cliper poco a poco ha ido ganando cada vez más clientes y, además, clientes grandes, sólo para tener una idea, Cliper realiza mantenimiento a los aires acondicionados de TELMEX, que es la compañía de telefonía más grande de México, también a Telefónica México, entre otros clientes empresariales de gran importancia. Escalar el sistema actual representa costos que la empresa debe asumir en contratar personal capacitado para extender el sistema actual, el cual, como se dijo anteriormente, sólo una persona en la empresa lo conoce técnicamente a profundidad. 

OpenERP es un sistema de gestión empresarial y contable de código abierto, disponible de manera gratuita en internet y en el que trabajan cientos de desarrolladores a nivel mundial, lo que garantiza por un lado estabilidad y escalabilidad, pues no es sólo una persona en el mundo quien conoce la herramienta técnicamente. Además, al ser de código abierto, garantiza adaptabilidad y una mayor transparencia en lo que es capaz de hacer la herramienta pues se tiene acceso al código fuente, y al estar disponible en internet, ahorra costos de licenciamiento en los que se incurre al utilizar tecnología basada en software privativo, es por ello que se planteó la opción de implementar la solución final utilizando OpenERP como plataforma ya que, al estar desarrollado en Python, también sigue la misma filosofía modular para ser extendido funcionalmente por lo que desarrollar módulos que añadan funcionalidad a OpenERP según las necesidades de negocio de un cliente en particular, resulta relativamente fácil. Mediante la implementación de OpenERP, se busca automatizar el proceso de facturación de contratos de mantenimiento, además de resolver los problemas de escalabilidad del sistema actual, cambiándolo por uno más estándar.

\section{Objetivos}
\subsection{Objetivo General}
\begin{itemize}
\item Desarrollar un Sistema de Gestión para Pólizas de Mantenimiento de Equipos de Refrigeración utilizando el Framework OpenObject
\end{itemize}

\subsection{Objetivos Específicos}
\begin{itemize}
\item Analizar el modelo de datos actual y modificarlo para aprovechar los modelos del Framework.
\item Adaptar los modelos del Framework a las necesidades del proyecto.
\item Crear los modelos nuevos, particulares del proyecto, para la extensión del Framework.
\item Generar la lógica para la facturación automatizada a partir del Sistema de Gestión de Pólizas.
\item Gestionar el esquema de Costos con Contabilidad Analítica.
\item Realizar la migración de los datos al nuevo modelo.
\item Generar los reportes y formatos necesarios para el análisis de los datos del sistema.
\end{itemize}
\newpage

\section{Descripción de la Empresa}
\lhead{Descripción de la Empresa}

%TODO: Preguntarle a Nhomar o a humberto el fin de semana

\newpage

\section{Metodología Empleada}
\lhead{Metodología Empleada}
La empresa Vauxoo, implementa una metodología de desarrollo bajo el esquema del Desarrollo Ágil, las metodologías ágiles son una forma novedosa de desarrollar software pues son metodologías que no pretenden predecir todo lo que va a ocurrir durante el desarrollo de un proyecto sino que, por el contrario, están diseñadas para adaptarse a los cambios lo mejor posible. Los principios del desarrollo ágil fueron enunciados por Kent Beck en el año 2001, en un documento llamado ``El Manifiesto Ágil" (The Agile Manifesto) el cual, se cita a continuación:

``Estamos descubriendo nuevas maneras de desarrollar software tanto por nuestra propia experiencia como ayudando a terceros. A través de esta experiencia hemos aprendido a valorar:

\begin{itemize}
\item \textbf{Individuos e interacciones} sobre procesos y herramientas.
\item \textbf{Software que funciona} sobre documentación exhaustiva.
\item \textbf{Colaboración con el cliente} sobre negociación de contratos.
\item \textbf{Responder ante el cambio} sobre el seguimiento de un plan.
\end{itemize}

Esto es, aunque los elementos a la derecha tienen valor, nosotros valoramos por encima de ellos a los que están a la izquierda".

Existes varias metodologías enfocadas al esquema ágil, Vauxoo implementa Scrum, la cual se describe a continuación:

\subsection{SCRUM}

SCRUM es una metodología creada por Jeff Sutherland y su equipo de desarrollo a principios de la década de 1990 [14]. Los principios de SCRUM están enmarcados dentro del Manifiesto Ágil y es un proceso que lleva el Desarrollo de Software a través de las siguientes actividades: requerimientos, análisis, diseño, evolución y entrega. Cada una de esas actividades son realizadas dentro de un patrón de trabajo llamado Sprint, todo el trabajo realizado dentro de un Sprint es adaptado al problema y frecuentemente modificado por el equipo de desarrollo a medida que las condiciones van cambiando.

SCRUM hace énfasis en la utilización de procesos de software que son efectivos en proyectos con tiempos cortos de entrega y requerimientos cambiantes. Esos procesos, en general, definen dos grandes actividades de desarrollo:

\begin{itemize}
\item \textbf{Backlog:} el backlog, constituye una lista priorizada de requerimientos que agregan valor de negocio al producto, estos requerimientos pueden ser agregados en cualquier momento y, de esta manera, se introducen los cambios en el proyecto [14]. El backlog puede ser, además, modificado en cualquier momento para adaptar las prioridades a los cambios del negocio.
\item \textbf{Sprints:} los Sprints constituyen paquetes de trabajo que son necesarios para desarrollar un requerimiento o un conjunto de ellos [14]. Los cambios no pueden ser introducidos dentro de un Sprint, de esta manera se asegura que el trabajo que se realiza es estable, pues los requerimientos que se seleccionan para ser desarrollados en un Sprint ya deben estar definidos y se debe tener la certeza de que, en caso de cambiar, será manejable.
\item \textbf{Scrum Meetings:} son reuniones cortas que, usualmente, se realizan todos los días durante un sprint [14]. Durante los Scrum Meetings se responden a tres preguntas básicas:
\begin{itemize}
\item ¿Qué has hecho desde la última reunión?.
\item ¿Cuáles obstáculos has encontrado?.
\item ¿Cuál es tu planificación hasta la próxima reunión?.
\end{itemize}
\item \textbf{Demos:} se van entregando los resultados de las funcionalidades desarrolladas al cliente para que puedan ser evaluados [14]. De esta manera, se va entregando Software Listo y Funcional que agrega valor al negocio del cliente en cada iteración.
\end{itemize}

De esta manera, SCRUM permite que los equipos trabajen de manera eficiente y estable en proyectos en los que la incertidumbre siempre está presente, además, resulta ideal para proyectos de pasantía ya que el desarrollador está en constante contacto con el cliente, además de que el tutor empresarial, lleva control de los tópicos que han detenido al pasante durante el desarrollo para darle soporte y guiarlo para resolver lo más pronto posible.

\section{Desarrollo}
\lhead{Desarrollo}
Previo al desarrollo del proyecto que se referencia en este informe, el pasante ya llevaba alrededor de un mes trabajando para Vauxoo, por lo que ya tenía dominio básico de Python como lenguaje de programación, OpenObject como RAD Framework y OpenERP como herramienta, por ello, realizar una inducción durante el período de ocho semanas de pasantía resultó innecesario
\end{document}
