\documentclass[a4paper, 12pt]{article}
\usepackage[utf8]{inputenc}
\usepackage[spanish]{babel}
\usepackage{fancyhdr}
\usepackage{setspace}
\usepackage[scaled]{helvet}
\usepackage{makeidx}

\setlength{\parskip}{\baselineskip}

\pagestyle{fancy}
\lhead{}
\chead{}
\rhead{Israel Fermín Montilla}
\lfoot{}
\cfoot{\thepage}
\rfoot{}
\renewcommand{\headrulewidth}{0.4pt}
\renewcommand{\footrulewidth}{0.4pt}

\begin{document}
\lhead{Índice}
\tableofcontents

\newpage

\section{Sinopsis}
\lhead{Sinopsis}
El presente informe, describe todos los detalles de la pasantía realizada por el Br. Israel Fermín Montilla en la empresa Vauxoo, desarrollando un Módulo para la herramienta de Planificación de Recursos Empresariales (ERP) llamada OpenERP, utilizando OpenObject como marco de trabajo y el lenguaje de programación Python, para la empresa mexicana Clima Perfecto (Cliper), la cual se dedica a proveer servicio de Mantenimiento e Instalación de equipos de Aire Acondicionado y Refrigeración, a través de contratos de póliza. El módulo desarrollado, debía soportar el proceso de facturación automática para dichos contratos.

El desarrollo del módulo surge de la necesidad en Cliper de cambiar su actual sistema ERP, el cual es un desarrollo interno basado en SQL Server 2005. El modelo de datos usado por el sistema antígüo será mostrado en el desarrollo de este informe. Este sistema, sólo provée información acerca de las pólizas activas mas no automatiza los procesos como tal. Con la migración a OpenERP, de busca mejorar el estado actual en cuanto a Sistemas Administrativos en Cliper, mediante la automatización del sistema de contabilidad, administración de recursos humanos y gestión automática de pólizas de mantenimiento de aires acondicionados, si bien la situación no se puede considerar "mala" o "deficiente" en cuanto a tecnología, pues si existía un sistema para gestionar la información de clientes, proveedores, personal y contratos, esta se podía mejorar mediante la automatización de procesos (o subprocesos) críticos de la empresa y esto es exactamente lo que se desea lograr a través de OpenERP.

En este informe se describirá todo el desarrollo del módulo de gestión y facturación automática para pólizas de manteminiento de aires acondicionados y equipos de refrigeración, desde el comienzo, hasta la entrega final del módulo. 

\newpage
\section{Planteamiento del Problema}
\lhead{Planteamiento del Problema}



\end{document}
